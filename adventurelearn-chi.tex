% {{{ preamble 
%%
%% Commands for TeXCount
%TC:macro \cite [option:text,text]
%TC:macro \citep [option:text,text]
%TC:macro \citet [option:text,text]
%TC:envir table 0 1
%TC:envir table* 0 1
%TC:envir tabular [ignore] word
%TC:envir displaymath 0 word
%TC:envir math 0 word
%TC:envir comment 0 0
%%
%% Switch the documentclass for review vs camera-ready etc
%\documentclass[manuscript,screen,review]{acmart}
\documentclass[manuscript,review,anonymous]{acmart}

%% \BibTeX command to typeset BibTeX logo in the docs
\AtBeginDocument{%
    \providecommand\BibTeX{{%
        \normalfont B\kern-0.5em{\scshape i\kern-0.25em b}\kern-0.8em\TeX}
}}

%% Rights management information.
\setcopyright{acmcopyright}
\copyrightyear{2021}
\acmYear{2021}
%\acmDOI{10.1145/1122445.1122456}

%% These commands are for a PROCEEDINGS abstract or paper.
\acmConference[CHI '21]{CHI '21: ACM Conference on Human Factors in Computing Systems}
{April 30--May 06, 2021}{New Orleans, LA}
\acmBooktitle{CHI '21: ACM Conference on Human Factors in Computing Systems, April 30--May 06, 2021, New Orleans, LA}
%\acmPrice{15.00}
%\acmISBN{978-1-4503-XXXX-X/18/06}

%% Submission ID.
%% Use this when submitting an article to a sponsored event. You'll
%% receive a unique submission ID from the organizers
%% of the event, and this ID should be used as the parameter to this command.
%%\acmSubmissionID{123-A56-BU3}

% }}} Preamble

%% end of the preamble, start of the body of the document source.
\begin{document}

%% The "title" command has an optional parameter,
%% allowing the author to define a "short title" to be used in page headers.
\title{Exploring Gamification for Understanding Learning Behaviors}

% {{{ Authors
%%
%% The "author" command and its associated commands are used to define
%% the authors and their affiliations.
%% Of note is the shared affiliation of the first two authors, and the
%% "authornote" and "authornotemark" commands
%% used to denote shared contribution to the research.

\author{Chek Tien Tan}
%\authornote{Both authors contributed equally to this research.}
\email{chektien.tan@singaporetech.edu.sg}
\orcid{0000-0002-1023-8819}

\author{Leon Foo}
%\authornotemark[1]
\email{leon.foo@singaporetech.edu.sg}

\author{Adriel Yeo}
\email{adriel.yeo@singaporetech.edu.sg}

\author{Jeannie S. Lee}
\email{jeannie.lee@singaporetech.edu.sg}

\author{Gideon Cheong}
\email{gideon.cheong@singaporetech.edu.sg}

\author{Edmund Wan}
\email{edmund.wan@singaporetech.edu.sg}

\author{Kenan Kok}
\email{kenan.kok@singaporetech.edu.sg}

\affiliation{%
  \institution{Singapore Institute of Technology}
  \streetaddress{10 Dover Drive}
  %\city{Singapore}
  %\state{Singapore}
  \country{Singapore}
  \postcode{138683}
}

% Independent Author
%\author{Charles Palmer}
%\affiliation{%
  %\institution{Palmer Research Laboratories}
  %\streetaddress{8600 Datapoint Drive}
  %\city{San Antonio}
  %\state{Texas}
  %\country{USA}
  %\postcode{78229}}
%\email{cpalmer@prl.com}

%% By default, the full list of authors will be used in the page
%% headers. Often, this list is too long, and will overlap
%% other information printed in the page headers. This command allows
%% the author to define a more concise list
%% of authors' names for this purpose.
\renewcommand{\shortauthors}{Tan, et al.}

% }}}

% {{{ Abstract and Keywords
%% The abstract is a short summary of the work to be presented in the
%% article.
%% 150 words max
\begin{abstract}
Navigating large-scale virtual spaces is a major challenge in VR applications due to real-world spatial limitations.
Walking in place (WIP) is a class of VR locomotion techniques that aims to address this challenge. 
Although WIP conceptually provides a naturalistic solution that shares similar qualities with real-world walking, there is little knowledge as to how WIP can be implemented in a manner that is usable, immersive and comfortable.
This paper hence provides some design guidelines for WIP by evaluating three common WIP implementation approaches (Head Bob, Lef Lift and Arm Swing) with consumer head-mounted displays (HMDs) with 41 participants.
We found that Leg Lift elicited the most positive experiential responses from a corroborated analysis of think-aloud and interview data, supplemented with self-reports of presence, flow and cybersickness.
\end{abstract}

%%
%% The code below is generated by the tool at http://dl.acm.org/ccs.cfm.
%% Please copy and paste the code instead of the example below.
%% Please use the 2012 Classifiers and see this link to embed them in the text: \url{https://dl.acm.org/ccs/ccs_flat.cfm}
%%
\begin{CCSXML}
<ccs2012>
<concept>
<concept_id>10003120.10003121</concept_id>
<concept_desc>Human-centered computing~Human computer interaction (HCI)</concept_desc>
<concept_significance>500</concept_significance>
</concept>
<concept>
<concept_id>10003120.10003121.10003125.10011752</concept_id>
<concept_desc>Human-centered computing~Haptic devices</concept_desc>
<concept_significance>300</concept_significance>
</concept>
<concept>
<concept_id>10003120.10003121.10003122.10003334</concept_id>
<concept_desc>Human-centered computing~User studies</concept_desc>
<concept_significance>100</concept_significance>
</concept>
</ccs2012>
\end{CCSXML}

\ccsdesc[500]{Human-centered computing~Human computer interaction (HCI)}
\ccsdesc[300]{Human-centered computing~Haptic devices}
\ccsdesc[100]{Human-centered computing~User studies}

% Author Keywords
\keywords {Virtual Reality, Walking-In-Place, Locomotion, Immersion}

%% This command processes the author and affiliation and title
%% information and builds the first part of the formatted document.
\maketitle

% }}}

% {{{ Introduction
\section{Introduction}

% background

% establish problem in navigating large virtual spaces

% controller-based locomotion: the brute force solution
A straightforward implementation for VR locomotion is controller-based locomotion, but suffers from the shortcoming of inducing intense cybersickness. Controller-based locomotion allows virtual navigation via directional pads/sticks and/or buttons with video game-style controllers. Although this works for traditional desktop or handheld game experiences, it is almost unusable in VR settings due to excessive cybersickness~\cite{controllerCybersickness2014}. The most cited theory to explain such cybersickness occurrences is the visual-vestibular conflict~\cite{visualVestibularConflictVR2020} which can be reasonably applied here as there is a strong mismatch between visual stimuli (virtual movement) and real life motion (being stationary with controllers).

% teleportation: the most common stop-gap solution
Virtual teleportation is a popular locomotion method in commercial VR titles although it may not be suitable for VR applications that require naturalistic locomotion. Teleportation has been shown to reduce cybersickness when compared to controller-based locomotion~\cite{teleportLessCybersickness2017} and is a prominent locomotion method even in recent VR game titles by major publishers, e.g., Valve Corporation's \emph{Half-life: Alyx}~\cite{halfLifeAlyxonSteam2021}. However, the notion of teleportation itself is entirely fictional and may not fit in virtual experiences that prioritize naturalistic locomotion, e.g., in a commuting simulation.

% commercial solutions fall short, expensive bulky, unusable
Commercially, several mechanical hardware solutions exist for VR locomotion but currently exists in obstrusive form factors and requires costly add-on hardware. Stationary VR ``treadmills" (e.g., KAT VR \cite{KATVROfficialSite2021}, Virtuix Omni~\cite{virtuixOmniOfficialSite2021} and Cyberith Virtualizer~\cite{cyberithVirtualizer2021}) inhibits accessibility by requiring users to purchase expensive and bulky hardware, and have questionable usability in which they require users to wear custom shoe covers to enable sliding of their feet on a slippery surface whilst suspended by a harness. This seemingly obtrusive form of interaction had the community sometimes referring to such VR hardware as ``slidemills" instead~\cite{slidemillsAniwaablog2018}. Actual belt-driven treadmills exist but mostly appear to be in constant development phase and not made available to the general public~\cite{infinadeck2021,spaceWalkerVR2021,aperiumRealityK01Pod2021}.

% need a cheap and accessible solution
Motivated by the lack of viable usable and accessible solutions for naturalistic VR locomotion, this paper aims to inform the implementation of such locomotion methods by comparing various low-cost and easy-to-setup walking-in-place implementation approaches. To establish the scope of the study chosen in this paper, a review of the state of research in VR locomotion methods is presented next.

% }}}

% {{{ Literature Review 
\section{Review of VR Locomotion Methods}

\subsection{Walking-in-place}

Walking-in-place~\cite{virtualLoco1999} is popular method that has been somewhat explored in prior research.

Prior work has compared the locomotion performances of users for Arm Swing and Leg Lift methods~\cite{vrLocoWalkingHeadBob2016} and found that Leg Lift produced the best spatial awareness and distance estimation performances. However they did not evaluate experiential constructs like presence and cybersickness.

Arm Swing is a common implementation approach for WIP. For example, Myo arm~\cite{mccullough2015myo} uses a wearable EMG Myo armband to detect arm swinging and found that

Other than walking-in-place, other methods like custom mechanical shoes~\cite{podoportation2020}, custom locomotion chairs and ... have been explored. However, these methods are in their infancy and the ease to recreate the hardware setup limits the accessibility to the common consumer.

This paper hence aims to limit the scope of investigation to locomotion methods that only rely hardware components that are easily accessible to a VR consumer, namely, consumer-grade HMDs (e.g., HTC Vive Pro) itself and consumer-grade tracking devices (e.g., the Vive Tracker add-ons).

% }}}

% {{{ Method
\section{Method}

A user test comparing 4 locomotion methods, namely arm swinging, leg lifting, head bobbing and a combination of arm swinging and leg lifting was conducted by the team. We evaluated the presence, flow state and simulation sickness levels of these methods in a within-subjects methodology with randomized sequence to maintain a counter balanced result.

\subsection{Implementation}
4 different locomotion method were implemented and compared for the feasibility of walking in infinite space. They are the following:
Arm-Swing, Head-Bob, Leg-Lift and a combination of Arm-Swing and Leg-Lift.

\subsubsection{Arm-Swing}
The algorithm is based on Arm Swing Paper~\cite{armSwing2017} by Yun Suen Pai and Kai Kunze. However, the algorithm on correcting the direction is modified from correcting direction based on previous direction outputs into increasing the number of direction inputs in-order to correct output. This is done due to the lack of clarity on smoothing or correcting of the output in the Arm Swing Paper and to retain good quality for the direction output without the limitation of moving for a while first.

For the VR arm swing locomotion, no additional input is needed except for the velocity from each controller and the facing direction given from the headset. Do note that the facing direction do not directly affect the direction that the user is moving toward in the virtual environment but it is used to determine the forward and backward direction of the user arm swing which will be further explained below.

The brief concept of the algorithm is to use each hand velocity and determine if a swing motion of forward or backward occur for that hand. The swing motion data is then passed to a controller that determines if the user is to move in the virtual environment.

For the magnitude test is to check the velocity of hand against a set amount to remove noise input caused by a slight motion of the hand.

The direction test is to determine if the current hand swing is forward or backward. The determining factor is to use the velocity vector from the hand to project onto the facing direction vector to evaluate the hand velocity vector is aligning( forward ) or against ( backward ) the facing direction. The projection method is done by using dot product of the hand velocity vector with the facing vector.

Forward: \[ \displaystyle Vhand \cdot Vfacing >= 0\]

Backward: \[ \displaystyle Vhand \cdot Vfacing < 0\]

The error count serves as the purpose to make sure that any false input or noise does not affect the input set.

To determine that an actual swing has occurred, the algorithm uses an input set to ensure that the user is swinging his hand as well as to gain a set of velocity that can be used for averaging to get a better quality velocity data. However, the amount of input needed to determine as a swing motion is based on testing with two factors in mind which are how natural the walk in the natural environment and the quality of velocity data. 

Using the velocity data, it is then used to calculate the walking direction in the Virtual Environment.

The hand manager has two processes which are to have a breather time to take in both hand data and register them as a swing motion to serve as a receiver and generate the walking direction in the virtual environment.

The calculation of the walking direction is done by :

\[ \displaystyle WalkDirection = N(Vforward) - N(Vbackward)\]
where N() is normalized function

The speed of the user is walking is based on the average magnitude from both hand velocity and apply onto min-max function with a set multiply amount.


\subsubsection{Head-Bob}
The algorithm used is based on the paper~\cite{headBob2018} by Juyoung Lee, Sang Chul Ahn and Jae-In Hwang. Step recognition is performed through the tracking of the position and orientation of the HMD. Each step is then translated into virtual velocity which is used to move the user in VR.

In the paper, calibration was done before hand to determine the range of a step to prevent unintentional movement. To further ensure that the range is always following the user's head, the range calculation and calibration was done before each step recognition phase in the implementation of the HeadBob algorithm.


%\subsubsection{Calibration of Central Axis H}
%\[ \displaystyle H = \left\{ \begin {array}{rcl}H_{Init} + C_{up} \sin ( X_{rot} \frac{2 \pi} { 360 ^\circ }) ) 
%						& \mbox{,} & 0^\circ \leq X_{rot} < 90^\circ \\
%						H_{Init} + C_{down} \sin ( X_{rot} \frac{2 \pi} { 360 ^\circ }) ) 
%						& \mbox{,} & -90^\circ < X_{rot} < 0^\circ
%						\end{array}\right.\]

%where $H_{Init}$ is the initial $Y_{pos}$ of the user(height), $X_{rot}$ is the X rotation of the user and $C_{up}$ and $C_{down}$ are arbitrary values, 0.06 and 0.13 respectively.

%\subsubsection{Calculation of Range R}
%\[ \displaystyle R = H \pm oSpacing\]

%where $H$ is the central axis and $oSpacing$ is an arbitrary value, 0.08 and 0.13 respectively.

%\subsubsection{Step Recognition}
%If $Y_{pos}$ of the HMD is inside $R$, we add the values into a moving average filter of $k$.
%The, we check the central datum among the inserted $n$ data to determine if its the lowest and therefore is a step.

%\subsubsection{Virtual Velocity}
%We check  the central datum again among the inserted $n$ data to determine if it's the highest, giving us the highest peak of the data. This results in us obtaining $S_{min}$ and $S_{max}$ as well as $s$ which is the highest and lowest peak of the data and the difference respectively. From there, we can obtain the initial speed from the following:

%\[   V_{0} = \left\{ \begin {array}{rcl}V_{min} + \frac{(s-S_{min})(V_{max}-V_{min})}{S_{max}-S_{min}}
%& \mbox{,} & ( S_{min}< s \leq S_{max}) \wedge (I_{min} < i_{step} \leq I_{max}) \\
%						0 & \mbox{,} & else
%						\end{array}\right.\]

%where $V_{min}$ and $V_{max}$ is the min and max velocity the user can move, $S_{min}$, $S_{max}$ and $s$ is the highest and lowest peak of the data and the difference respectively and $I_{min}$, $I_{max}$ and $i_{step}$ is the min and max time a step can happen and the time taken for the step.

%From $V_{0}$, we can calculate the final velocity with the time elapsed, $i_{step}$

%\[ v = V_{0} - a * i_{step} \]

%where a is the acceleration of the user.

\subsubsection{Leg-Lift}
The algorithm used is based on the paper~\cite{legLift2008} by Jeff Feasel, Mary C. Whitton and Jeremy D. Wendt. Using trackers attached to the user's legs, the heel position is used to obtain the vertical velocity by performing a numeric differentiation on the vertical component. The absolute value of the vertical velocity, which will be discarded if it is to huge or too small, is then passed through a Butterworth Filter with a cutoff frequency of 5Hz. Lastly, a fixed value is subtracted from the result of the Butterworth filter to reduce on virtual locomotion drift. This results in a virtual locomotion speed that we apply it in virtual reality.

\subsection{Combination of Arm-Swing and Leg-Lift}

%\begin{figure*}
	%\centering
	%\includegraphics[width=1.0\textwidth]{figures/FlowDiagram}
	%\caption{The flow diagram of the locomotion system}
%\end{figure*}
%\subsection{Instrumentation}


\subsection{Participants}

\subsection{Procedure}

% }}}

% {{{ Results
\section{Results}

% }}} 

% {{{ Discussion
\section{Discussion}

\subsection{Insights for implementing WIP for VR locomotion}

% }}}

% {{{ Conclusion
\section{Conclusion}

% }}}

% {{{ Template snippets from sample file
%%
%% The majority of ACM publications use numbered citations and
%% references.  The command \citestyle{authoryear} switches to the
%% "author year" style.
%%
%% If you are preparing content for an event
%% sponsored by ACM SIGGRAPH, you must use the "author year" style of
%% citations and references.
%% Uncommenting
%% the next command will enable that style.
%%\citestyle{acmauthoryear}

%A ``teaser figure'' is an image, or set of images in one figure, that
%are placed after all author and affiliation information, and before
%the body of the article, spanning the page. If you wish to have such a
%figure in your article, place the command immediately before the
%\verb|\maketitle| command:
%\begin{verbatim}
  %\begin{teaserfigure}
    %\includegraphics[width=\textwidth]{sampleteaser}
    %\caption{figure caption}
    %\Description{figure description}
  %\end{teaserfigure}
%\end{verbatim}

%\section{Appendices}

%If your work needs an appendix, add it before the
%``\verb|\end{document}|'' command at the conclusion of your source
%document.

%Start the appendix with the ``\verb|appendix|'' command:
%\begin{verbatim}
  %\appendix
%\end{verbatim}
%and note that in the appendix, sections are lettered, not
%numbered. This document has two appendices, demonstrating the section
%and subsection identification method.

%%
%% The acknowledgments section is defined using the "acks" environment
%% (and NOT an unnumbered section). This ensures the proper
%% identification of the section in the article metadata, and the
%% consistent spelling of the heading.
%\begin{acks}
%To Robert, for the bagels and explaining CMYK and color spaces.
%\end{acks}

% }}}

% {{{ Bibliography
%% The next two lines define the bibliography style to be used, and
%% the bibliography file.
\bibliographystyle{ACM-Reference-Format}
\bibliography{bib.bib}

% }}}

% {{{ Appendices
%%
%% If your work has an appendix, this is the place to put it.
%\appendix

%\section{Research Methods}

%\subsection{Part One}

%Lorem ipsum dolor sit amet, consectetur adipiscing elit. Morbi
%malesuada, quam in pulvinar varius, metus nunc fermentum urna, id
%sollicitudin purus odio sit amet enim. Aliquam ullamcorper eu ipsum
%vel mollis. Curabitur quis dictum nisl. Phasellus vel semper risus, et
%lacinia dolor. Integer ultricies commodo sem nec semper.

%\subsection{Part Two}

%Etiam commodo feugiat nisl pulvinar pellentesque. Etiam auctor sodales
%ligula, non varius nibh pulvinar semper. Suspendisse nec lectus non
%ipsum convallis congue hendrerit vitae sapien. Donec at laoreet
%eros. Vivamus non purus placerat, scelerisque diam eu, cursus
%ante. Etiam aliquam tortor auctor efficitur mattis.

%\section{Online Resources}

%Nam id fermentum dui. Suspendisse sagittis tortor a nulla mollis, in
%pulvinar ex pretium. Sed interdum orci quis metus euismod, et sagittis
%enim maximus. Vestibulum gravida massa ut felis suscipit
%congue. Quisque mattis elit a risus ultrices commodo venenatis eget
%dui. Etiam sagittis eleifend elementum.

%Nam interdum magna at lectus dignissim, ac dignissim lorem
%rhoncus. Maecenas eu arcu ac neque placerat aliquam. Nunc pulvinar
%massa et mattis lacinia.

% }}}

\end{document}
\endinput
